% Introdu��o
\chapter{Outliers}


Outliers in the statistics area are when one finds "aberrant values" in a given 
series of data, that is when one finds an atypical value or with a great distance 
from the normal distribution in that set. For example, when a researcher wants to 
monitor the temperature of his CPU during a certain time interval and it has been 
realized that the average temperature range is between 34 ºC and 48 ºC degrees
being 45 ºC the maximum temperature and 27 ºC the minimum temperature and in the 
middle of this sample are some punctual registers of 0 ºC, this can be characterized
as an outlier and, most likely, will be understood as a malfunction of the equipment
that performed the collection of these CPU temperatures.

However, there are several ways to interpret an Outlier (not only as a collection 
error), but also as: data that belong to a different population of the sample, a 
damaged data, areas in which a certain theory is not valid or even, when the sample 
is too large, it is normal to have some small amounts of outliers in that group.
In cases where it is proven that it is not the fault of a collection equipment 
malfunction or that it was not a human mistake, it is extremely important to know 
the why of that outlier and try to understand it, because it is not interesting for 
a research simply remove it from the sample or re-signify it by assigning a new 
value. This change may compromise the validity of the research, and if this is 
done, it is extremely important to document and record those changes.