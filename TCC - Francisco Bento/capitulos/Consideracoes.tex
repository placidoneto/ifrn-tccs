% Considera��es finais
\chapter{Considerações finais}
\label{chap:consideracoes}

Neste trabalho é proposto uma nova versão da plataforma GeoGuide \cite{omidvarTehrani2017} que leva em consideração os conceitos de IDR e a detecção de outliers para aprimorar a análise do usuário e as sugestões de novos pontos pela plataforma e com isso trazer mais opções sobre como o analista pode observar o dataset em busca de informações implícitas e quais os possíveis próximos passos para descobrir mais conhecimentos sobre aquele dataset em análise pelo usuário.

\section{Principais contribuições}

O foco deste trabalho é acrescentar novos fatores para a abordagem de orientação do GeoGuide: os IDRs e a detecção de outliers. Com isso, a preferência do usuário por determinadas regiões podem agora ser levada em consideração e também o processo de encontrar possíveis anomalias, e pontos que mereçam uma atenção mais específica para se compreender melhor, se tornou mais interativo e dinâmico, facilitando o uso por parte do usuário e dinamizando o processo iterativo de exploração do dataset.

\section{Trabalhos futuros}

Como trabalhos futuros é sugerido uma melhoria no processo para calcular as métricas base do GeoGuide, também pode ser adicionado uma funcionalidade para o GeoGuide em que se leve em consideração múltiplos datasets e assim, com múltiplos contextos, será facilitado o processo para encontrar as razões pelas quais determinados outliers estão presentes naquele conjunto e assim dinamizar a exploração de vários datasets em um mesmo contexto. 

Além disso é indicado a exibição das melhorias propostas em usos práticos dos \textit{datasets} citados, com exemplos reais de como essas melhorias ajudaram o usuário no processo de exploração de determinados conjuntos de dados.