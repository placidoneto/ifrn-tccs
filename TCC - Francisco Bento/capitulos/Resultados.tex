\chapter{Resultados}

Neste capítulo nós mostraremos quais os resultados que obtivemos com essa nova proposta do GeoGuide levando em consideração os conceitos de IDRs e detecção de outliers para a melhoria da análise do usuaŕio e no sistema de recomendação de pontos pela plataforma.

\section{Experimentos}

TODO: Dar dois exemplos e experimentos para conseguir mostrar alguma utilização dessa abordagem com outliers e IDRs.

TODO: Falar do primeiro exemplo, utilizando a conta na platorma, registrando o dataset do yelp, dizendo quais os campos que vai levar em consideração, abrir o mapa, navegar pelas regiões [se der, pegar um print mostrando as IDRs geradas com os outliers de cada IDR], escolher um ponto específico e pedir as sugestões.

TODO: quando receber as sugestões, mostrar o outlier, baseado na IDR, de uma forma diferente (cor ou ícone diferente) ou então mostrar um ponto de atributo anormal e explicar como ele pode ter sido sugerido baseado no algoritmo de detecção de outlier.

TODO: Finalizando o primeiro exemplo, explicar a importância dessa estratégia e quais são as vantages e o que se pode retirar de informações sobre isso. Também falar dos possíveis contras e do que se pode melhorar.



TODO: Segundo exemplo, falar seguindo a mesma estrutura, mas utilizando o dataset do airbnb. Fluxo inicial extremamente parecido com o primeiro exemplo e depois partir pro resultado do GeoGuide. O ideal seria encontrar um outlier baseado no preço (sendo negativo ou positivo) e justificar sua importância para conseguir entender mais sobre o dataset.

TODO: Seguir o fluxo do primeiro exemplo e explicar mais uma vez as vantagens e possíveis problemas desse uso, já pensando em como solucionar no futuro.