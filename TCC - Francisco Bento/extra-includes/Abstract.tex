% Resumo em l�ngua estrangeira (em ingl�s Abstract, em espanhol Resumen, em franc�s R�sum�)
\begin{center}
	{\Large{\textbf{\thesistitle}}}
\end{center}

\vspace{1cm}

\begin{flushright}
	Author: \thesisauthor\\
	Supervisor: \thesissupervisor
\end{flushright}

\vspace{1cm}

\begin{center}
	\Large{\textsc{\textbf{Abstract}}}
\end{center}

\noindent With information technology advancement, the volume of spatial data generated has grown daily. These characteristics complicate the data  analysts investigation, since they may `` feel lost '' in the midst of a large amount of data. Besides that, these large volumes may contain data that are discrepant and hard to analyze. This paper presents a proposal of a new version of GeoGuide, a framework that aims assist the analyst during his exploration of large volumes of spatial data, taking into account the regions of user interest and the outlier data that are present in the set. This proposal was based on the perception that discrepant data present in a set of spatial data can compromise the analysis and impair the analyst's understanding of that dataset. This paper also discusses how this new approach works and its advantages based on real data examples.

\noindent\textit{Keywords}: Spatial data, User interest, Outliers.