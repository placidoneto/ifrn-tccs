% Resumo
\begin{center}
	{\Large{\textbf{\thesistitle}}}
\end{center}

\vspace{1cm}

\begin{flushright}
	Autor: \thesisauthor\\
	Orientador(a): \thesissupervisor
\end{flushright}

\vspace{1cm}

\begin{center}
	\Large{\textsc{\textbf{Resumo}}}
\end{center}

\noindent Com o avanço da tecnologia da informação, o volume de dados espaciais gerados tem crescido diariamente. Essa característica complica a exploração dos analistas de dados, pois eles podem se ``sentir perdido'' no meio de uma grande massa de dados. Além disso, esses grandes volumes podem conter dados discrepantes e de difícil análise. Neste trabalho é apresentada uma proposta de nova versão do GeoGuide, uma ferramenta que visa auxiliar o analista durante a exploração de grandes volumes de dados espaciais, que leva em consideração as regiões de interesse do usuário e os possíveis dados discrepantes que estejam no conjunto. Esta proposta foi fundamentada na percepção de que dados discrepantes presentes em um conjunto de dados espaciais podem comprometer a análise e prejudicar o entendimento do analista no que diz respeito aquele conjunto de dados. Também é tratado o funcionamento dessa abordagem e quais as vantagens baseadas em exemplos de dados reais.

\noindent\textit{Palavras-chave}: Dados espaciais, Interesse do usuário, Discrepância de dados.